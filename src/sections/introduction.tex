The following is a description of the
conditions, experience, and scientific gains
of the RAVEN fieldwork in Summer of 2022,
as it pertains to a Computer Science PhD student.
It is intended to show that computer science research need not occur only in
a typical indoor, office-based environment,
but can be cross-disciplinary and even relevant in Icelandic highland environments
that are principally of interest to geologists.
It also aims to show that the RAVEN fieldwork -- while difficult -- can present a good opportunity
for students in general to broaden their research horizons
and to network with other, leading researchers.

The Rover-Aerial Vehicle Exploration Network (RAVEN) is a project to evaluate the scientific gains
made by combining rover and drone teams for Mars exploration, instead of using each vehicle by itself.
It is a US-based Planetary Science and Technology from Analog Research (PSTAR) program,
that uses Iceland's rough terrain -- particularly Iceland's highlands and lava flows --
as its analog research environment.
It is headed by Christopher Hamilton and Joanna Voigt from the University of Arizona,
and also has collaboration from Reykjavik University, the University of Tennessee,
the University of Western Ontario, Honeybee Robotics, the Canadian Space Agency, and NASA.
It is a great opportunity for students from Reykjavik University to collaborate with such researchers.
